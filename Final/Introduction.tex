\section{Introduction}
  
  The Random Access Machine (RAM) model of computation is widely used for
  analyzing the performace of algorithms. Some, such as Jurkiewicz and Mehlhorn
  question whether this model accurately represents the complexities of modern
  hardware. They propose the Virtual Address Translation (VAT) model of
  computation to account for added complexities, such as the memory address
  translation needed to refference ever larger amounts of system memory in 
  modern computers.
  
  This paper analyzes a set of algorithms using both models, VAT and RAM,
  and presents the results of benchmarking these algorithms on real hardware.
  The goal of this is to then use the results of the analysis and benchmarks
  to determine which model more accurately reflects the real hardware.
  
  The novel contribution includes not only the benchmarking of these procedures,
  but also the comparative analysis of the two models. This work could be very
  significant if VAT is shown more accurate than RAM. We would be the first to
  independently verify this and we would be contributing some of the first
  analysis of common, reference algorithms. (We're still working on results.)

  This paper is organized as follows: Section 2 presents the paper's
  motivation. Section 3 addresses previous work. Section 4 details how the
  benchmarks and analyses were done. Section 5 presents the results. Section 6
  discusses the results, selecting the more accurate model. Section 7 and 8 
  address the potential for future work and conclude the paper.