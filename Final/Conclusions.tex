\section{Conclusions}
The aim of this paper was to experimentally compare two models of computation,
the Random Access Machine model and a new model devised by Jurkiewicz and
Mehlhorn that incorporates the costs of Virtual Address Translation. This paper
began by introducing the contenders, explained the motivation to look for a new
model of computation, and noted the relevant prior work. The paper continued by
describing a method for comparing the two models and presenting the results of
that work. Having measured the running times of a set of algorithms and
compared the results with the predictions of the models, this paper determined
that the incumbent RAM model is superior to the new VAT model.

While it would have been exciting to validate this brand new model, to
potentially contribute to a small revolution in the field, it is comforting to
know that the widely used RAM model has earned its place by
continually demonstrating its validity and utility.
