\section{Methods}
  
  Three components are necessary to determine which model of computation yields
  a more accurate analysis of the performance of algorithms running on real
  hardware. First, there must be implementations of algorithms running on real
  hardware. Running time benchmarks establish a ground truth for doing
  comparisons. Second, the algorithms must be analyzed according to the new VAT
  model. And third, the algorithms must be analyzed using the RAM model. The 
  authors selected five well-studied algorithms which have been previously
  analyzed with the RAM model.
  
  \subsection{Benchmarking}
  
  To generate a standard for comparison, the authors selected a set of five
  algorithms:Binary Search, Heapsort, Insertionsort, Quicksort, and Permute. 
  These five are well understood, have a variety of running times, and should
  be straightforward enough to analyze on the new model.
  
  The algorithms were coded in C++ and compiled with GCC and Make. They were
  run on a VirtualBox virtual machine, with Debian GNU/Linux for 32 bit
  x86 processors. The virtual machine was hosted by 64 bit Windows 7 on a 2.0
  GHz Core i7 processor. The physical machine has 8 GiB of memory and four
  processor cores, of which 2.5 GiB and 2 cores were allocated to the virtual
  machine. Find a link to the source code in the appendix.
  
  The benchmark program ran each procedure 10 times on each input size. The 
  smallest input to each procedure was an integer array of length 1. The input
  doubled in size after every tenth run. The maximum size input array to 
  Binary Search was 536870912 32 bit integers, the 2 GiB size reaching the 
  limits of the machine's memory capacity. The maximum size input array to 
  Insertionsort was 1048576 32 bit integers, the execution time on any larger
  input becoming excessive and overflowing the capacity of the timer. The other
  three procedures had maximum size input arrays of 268435456 32 bit integers,
  each array taking up 1 GiB of system memory.
  
  The benchmark program timed each run of each procedure on each size of input.
  The raw results are available at the link provided in the appendix. The next
  section of the paper provides a summary presentation.
