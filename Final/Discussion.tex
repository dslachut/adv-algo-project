\section{Discussion}

The results of the experiments come out very clearly from the data. This
section compares the RAM model to the benchmarks, then the VAT model to the
benchmarks, and lastly the VAT model to the RAM model to determine which model
is superior.

    \paragraph{RAM and Benchmarks}
    As noted in the previous section, the running times of Heapsort and
    Quicksort are consistent with $O(n\log(n))$ behaviour, Insertionsort with
    $O(n^2)$ behaviour, and Permute with linear behaviour. These results are
    the same as predicted by the RAM model. 
    
    \paragraph{VAT not Necessary}
    The results of the benchmarks are not consistent with Jurkiewicz and
    Mehlhorn. They presented, as part of the motivation for their new model, a 
    graph showing large differences between the RAM-predicted time complexities
    and the actual running times of a set of algorithms [5]. In particular,
    Permute, Heapsort, and BinarySearch were shown in that paper to
    significantly depart from the RAM model at input sizes near $2^{21}$.
    Benchmarks done on a virtual machine could show this discrepancy even more
    clearly, as the virtual address translation would need to be done on both
    the virtual and the physical machines.
    
    The benchmarks described above present a different story. This paper tests 
    those three algorithms at input sizes well above $2^{21}$ and finds no 
    noticeable departure from the RAM model which would necessitate a new model
    of computation. 
    
    \paragraph{Recommendation}
    This paper cannot recommend the VAT model for use at this time. The
    experimental evidence points to the continuing utility and validity of the
    venerable RAM model of computation. The VAT model is far more difficult to
    use than is the RAM model, while it simultaneously fails to give a more
    accurate prediction of the running times of the tested algorithms.
    Therefore, the RAM model should continue to be used by most analysts.
