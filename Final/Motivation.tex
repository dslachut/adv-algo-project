\section{Motivation}

Analysis of algorithms on abstract machines serves as a crucial element of
computer science, which allows incredible increases in the speed and utility
of computer programs. Until now, computer scientists have used the RAM and the
EM models to perform algorithmic analysis accurately. But Jurkiewicz and
Mehlhorn observe discrepancies with some experimental findings and attempt to
push forward the VAT model to account for these discrepancies. 

Before algorithm analysts shift to this new model, they must carefully verify
its claims and correctness, it's utility. This is where Jurkiewicz and
Mehlhorn's contribution falls short. Their paper does mention that experimental
findings tally with the theoretical predictions of the new VAT model, but it
does not clearly represent these findings. Additionally the number of test
cases is limited. So before scientists readily accept this new model for
theoretically analyzing the running time of algorithms the test set must be
broadened to verify and compare the experimental results with the model's
predictions. 

If it can be verifed by experimental comparison that the new VAT model is more
accurate, then this will serve as an incentive to researchers to use the
proposed model for more accurate estimations of running time of algorithms.
More accurate algorithm analysis will help maintain the pace of innovation
enjoyed by computer science for a generation.